\chapter{绪论}

\section{引言}
\subsection{流变学的核心研究内容}
流变学是研究物质在外力作用下变形和流动的科学,其研究对象涵盖了流体、软固体以及在特定条件下可以流动的固体。流变学的核心在于揭示材料的应力、应变和时间之间的内在关系,并通过本构方程(流变状态方程)对这些关系进行定量描述。流变学的研究不仅深化了对材料力学行为的理解,还为工程应用和科学研究提供了重要的理论基础。流变学的核心研究内容主要包括以下几个方面:
\begin{enumerate}[topsep = 0 pt, itemsep= 0 pt, parsep=0pt, partopsep=0pt, leftmargin=44pt, itemindent=0pt, labelsep=6pt, label=(\arabic*)]
	\item 材料的流动与变形行为:材料的流动与变形行为是流变学研究的核心内容之一。通过实验和理论模型,流变学揭示了材料在外力作用下的复杂力学行为。例如,蠕变现象(即在恒定应力下,材料的变形随时间逐渐增加)和应力松弛现象(即在恒定应变下,材料的应力随时间逐渐减小)是流变学中重要的研究对象。这些现象不仅反映了材料的时间依赖性行为,还为材料的长期性能评估提供了理论依据。此外,流变学还研究了材料的非线性力学行为,如屈服、塑性变形和断裂等,这些研究对于理解材料的宏观力学性能具有重要意义。
	\item	  本构方程的构建:本构方程是流变学中用于描述材料力学行为的数学工具,其核心在于建立应力、应变和时间之间的定量关系。对于牛顿流体,其本构方程基于牛顿黏性定律,即应力与应变率成正比。然而,对于非牛顿流体和软固体,其本构方程则更为复杂,通常需要考虑材料的非线性、黏弹性以及时间依赖性等特性。通过构建合理的本构方程,流变学能够对各种物理现象进行精确的数学描述,从而为工程设计和材料开发提供理论支持。
	\item  实验与模拟方法:流变学实验是研究材料流变性能的重要手段,常见的实验方法包括蠕变实验、应力松弛实验和动力试验等。这些实验能够直接测量材料在不同条件下的力学响应,为理论模型的验证和优化提供实验数据。近年来,随着计算模拟技术的发展,流变学研究逐渐从唯象模型向定量科学转变。微观实验技术(如X射线散射、中子散射)与计算模拟的结合,使得研究者能够在微观尺度上揭示材料的流变机制,从而推动流变学向更高精度和更深层次发展。
\end{enumerate}
\subsection{流变学应用方向}
流变学的研究方向广泛,涵盖了多个学科和领域,例如高分子流变学研究高分子材料的分子结构与其流变行为的关系,例如聚合物熔体和溶液的拉伸流变行为。生物流变学研究生物材料(如血液、肌肉)的流变特性,揭示生理和病理过程中的力学机制。地质流变学研究岩石、土壤等地质材料的流变行为,应用于地震预测、矿产资源开发等领域。工业流变学在材料加工、食品工业、化妆品和医药制造等领域,流变学用于优化工艺和产品性能。非牛顿流体力学研究不符合牛顿黏性定律的流体(如油漆、泥浆、血液)的流动特性。

\section{本构方程}
\subsection{线性本构方程}
\subsection{非线性本构方程}
\subsection{传统的本构方程的构建方法}

\section{机器学习理论}
\subsection{传统机器学习方法}
\subsection{神经网络与深度学习}
\subsection{注意力机制}
\subsection{生成式模型}

\section{深度学习应用于本构方程研究现状}
\subsection{纯数据驱动方法}
\subsection{引入物理约束的神经网络研究}

\section{本课题研究介绍}
\subsection{研究内容}
\subsection{创新之处}
\subsection{研究意义}



