\chapter{绪论}

\section{研究背景和意义}
\subsection{研究背景和意义}

关于Latex以及基于Latex写作的好处不再赘述。\LaTeX{}的入门资料推荐文献\parencite{_g}以及文献\parencite{_c}。

这里主要是想推荐一种“学术生态”,即利用各种工具展开科研工作,以达到事半功倍的效果。需要用到以下软件:
\begin{enumerate}[topsep = 0 pt, itemsep= 0 pt, parsep=0pt, partopsep=0pt, leftmargin=44pt, itemindent=0pt, labelsep=6pt, label=(\arabic*)]
	\item 	参考文献管理软件zotero\cite{_m}。很多人使用过endnote,但其实zotero也非常强大,强烈推荐。可到b站观看Struggle with Me出品的视频教程\cite{_k}入门(或其他最新教程,刚开始不推荐使用插件,会增加学习难度)。zotero自带pdf阅读器,也可以设置为使用其他阅读器。在zotero可以打开文件所在位置,故不推荐更改zotero的文件系统(尤其不推荐使用zotfile插件,事实上各种五花八门的插件增加了复杂性,实际上没有带来太多便利性)。理论上只需要包含文献元数据信息的bib文件(可以手动一篇一篇文章地收集)即可使用此模板,因此模板不依赖于任何参考文献管理软件,endnote用户或不使用参考文献管理软件的用户可以忽略本文zotero部分的讲解。
	\item	可截图获取文献中公式的软件mathpix\cite{_h}。在阅读别人的论文时,很可能需要把文章中的公式抄下来放到自己的笔记中,方便以后组会报告甚至论文中使用,这时使用mathpix可直接截图获取\LaTeX{}源码,非常方便。该软件普通邮箱注册可每月50次免费,学校邮箱可100次,若信用卡注册可1000次(最新情况是只能500次了,还要收费20美元,世界变化太快了)。注:随着mathpix的使用成本越来越高,免费次数越来越少,2023起已经不再推荐。目前开源/免费的替代工具为:。\href{https://www.simpletex.cn/}{SimpleTex}和\href{https://p2t.breezedeus.com/}{Pix2Tex}。目前SimpleTex性能比较好,免费但不开源,不排除未来收费的可能
	\item	TeXlive202x、TeXstudio,相当于开发环境和IDE。本模板是基于TeX的发行版TeXlive202x和编辑器TeXstudio进行的,百度这两个关键字分别安装。关于TeXstudio的使用(快捷键等)可另行查找资料。模板还支持更多ide,更多编译方式见GitHub首页readme.md。若在其他窗口打开了编译生成的pdf文件,记得关掉再编译,否则报错。TeXstudio的设置见第二章。
\end{enumerate}

本文的章节安排如下:

第一章,绪论。

第二章,模板简介。主要介绍各文件的内容。

第三章,常用环境。介绍论文写作中常用的环境,包括:图、表、公式、定理。基本涵盖了常用的命令。
\section{研究实验设计}
\subsection{tes1}
你也会做实验吗?
\begin{align}
	\frac{1}{2} \tag{1-1}                   \\
	\sigma & = \sqrt{\frac{1}{2}} \tag{1-2}
\end{align}
\subsection{tes2}
\section{研究进展}
%第三章,参考文献设置。本模板对旧版的改动主要是参考文献部分,本章将简单参考文献设置以及
%编译选项的设置等等。


