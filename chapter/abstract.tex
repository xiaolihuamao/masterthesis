\chapter{摘\texorpdfstring{\quad}{}要}
流变学是研究物质变形和流动的科学,流变学本构方程作为流变学的核心内容之一,是描述材料在外力作用下变形和流动行为的数学模型,用于研究材料的应力、应变、应变率之间的关系。这些方程广泛应用于描述复杂材料的力学行为,如聚合物、胶体、血液、泥浆等,帮助理解和预测材料的力学行为,从而优化工艺、提高产品质量和推动科学研究。传统本构方程的获取和研究方法主要包括实验测定、理论推导和数值模拟,但这些方法存在成本高、耗时长、适用范围有限、计算资源需求大等问题。

近年来,基于数据驱动的机器学习方法开始被应用于流变学本构方程的构建研究中。然而,纯数据驱动的方法缺乏物理约束,难以捕捉复杂的物理关系,如流变学中的非线性黏弹性等。鉴于此,本文采用神经网络(深度学习)方法对流变学本构方程进行建模和预测,并通过引入物理约束,提升模型的训练效果和可解释性。具体而言,本文使用门控循环单元循环神经网络(GRU)和物理信息神经网络(PINN)分别对经典本构方程的模拟数据和动态力学分析(DMA)的实验数据进行深度学习建模和预测。

首先,本文通过数值模拟方法获取Maxwell、Doi-Edwards、Giesekus模型的时间域应力应变模拟数据,并使用GRU对其进行建模预测。创新性地利用GRU的门控机制捕捉时间序列应力应变数据的记忆效应和应变历史依赖性。与传统深度前馈神经网络(DNN)相比,GRU在R²、MAE、MAPE各项预测指标上表现更优。结果表明,GRU能够泛化到不同的应变加载协议和变化历史,捕捉流变学中复杂的非线性关系。

随后,本文对真实DMA实验数据进行建模。本文采用物理信息神经网络(PINN),在损失函数中引入物理方程残差,为模型提供物理约束,并与数据损失共同训练模型。同时,针对实验中多个制备参数作为特征的稀疏性和分散性,本文使用注意力特征融合方法进行特征降维,进一步优化了模型效果。结果表明,PINN的预测效果各项指标均优于纯数据驱动的深度学习模型,而哈达玛特征融合和注意力特征融合方法有效解决了特征稀疏分散问题。

最后,本文使用条件变分自编码器(CVAE)对实验数据进行反向建模,通过特定的流变学性质数据生成实验制备参数,为高分子材料的设计提供辅助支持。
\keywordsCN{流变学;本构方程;门控循环单元;物理信息神经网络;条件变分自编码器}

\chapter{Abstract}
Rheology is the science that studies the deformation and flow of matter. Constitutive equations in rheology, as one of the core components of rheology, are mathematical models that describe the deformation and flow behavior of materials under external forces. They are used to study the relationships between stress, strain, and strain rate in materials. These equations are widely applied to describe the mechanical behavior of complex materials such as polymers, colloids, blood, and slurries, helping to understand and predict the mechanical behavior of materials, thereby optimizing processes, improving product quality, and advancing scientific research. Traditional methods for obtaining and studying constitutive equations mainly include experimental measurement, theoretical derivation, and numerical simulation. However, these methods have issues such as high cost, long time consumption, limited applicability, and high computational resource requirements.

In recent years, data-driven machine learning methods have begun to be applied to the construction of constitutive equations in rheology. However, purely data-driven methods lack physical constraints and struggle to capture complex physical relationships, such as nonlinear viscoelasticity in rheology. In light of this, this paper employs neural network (deep learning) methods to model and predict constitutive equations in rheology, and enhances the training effectiveness and interpretability of the models by introducing physical constraints. Specifically, this paper uses Gated Recurrent Unit (GRU) and Physics-Informed Neural Networks (PINN) to perform deep learning modeling and prediction on simulation data of classical constitutive equations and experimental data from Dynamic Mechanical Analysis (DMA).

First, this paper obtains time-domain stress-strain simulation data for the Maxwell, Doi-Edwards, and Giesekus models through numerical simulation methods and uses GRU to model and predict them. Innovatively, the gating mechanism of GRU is utilized to capture the memory effects and strain history dependence of time-series stress-strain data. Compared to traditional Deep Feedforward Neural Networks (DNN), GRU performs better in terms of R², MAE, and MAPE prediction metrics. The results show that GRU can generalize to different strain loading protocols and change histories, capturing complex nonlinear relationships in rheology.

Subsequently, this paper models real DMA experimental data. The Physics-Informed Neural Network (PINN) is employed, introducing physical equation residuals into the loss function to provide physical constraints for the model, and training the model jointly with data loss. Additionally, to address the sparsity and dispersion of multiple preparation parameters as features in the experiments, this paper uses attention feature fusion methods for feature dimensionality reduction, further optimizing the model's performance. The results indicate that PINN's prediction performance metrics are superior to those of purely data-driven deep learning models, and the Hadamard feature fusion and attention feature fusion methods effectively solve the issue of sparse and dispersed features.

Finally, this paper uses Conditional Variational Autoencoder (CVAE) for inverse modeling of experimental data, generating experimental preparation parameters through specific rheological property data, providing auxiliary support for the design of polymer materials.
\keywordsEN{Rheology; Constitutive Equations; Gated Recurrent Unit; Physics-Informed Neural Networks; Conditional Variational Autoencoder}