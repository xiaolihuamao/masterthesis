\chapter{摘\texorpdfstring{\quad}{}要}
流变学是研究物质变形和流动的科学,流变学本构方程作为其核心内容之一,是描述材料在外力作用下变形和流动行为的数学模型。传统本构方程的获取和研究方法主要依赖实验测定、理论推导和数值模拟,这些方法存在成本高、耗时长、难以泛化等问题,难以满足现代材料科学对高效、精确建模的需求。

近年来,基于数据驱动的机器学习方法开始应用于流变学本构方程的构建研究中,为解决传统方法的局限性提供了新思路。然而,当前的机器学习本构建模方法仍存在物理约束不足、未充分考虑流变学本构方程的历史依赖性、特征稀疏以及难以反推制备参数等问题。

针对上述问题,本文从时间域流变数据建模和频域数据建模两个方向开展研究。在时间域应力应变数据方面,本文创新性地采用物理信息门控循环单元(PI-GRU)进行时变本构数据的预测建模,利用GRU的门控机制有效捕捉流变学中的记忆效应和应变历史依赖性,同时引入物理残差作为损失函数的一部分,引入物理约束。相比传统深度前馈神经网络,PI-GRU在经典本构方程的数值模拟数据和真实流变学实验数据上均表现出更好的预测效果和泛化能力。

在频域数据方面,本文采用物理信息神经网络(PINN)对材料的制备参数与储存模量、损耗模量、损耗角正切等频域数据关系进行正向建模,通过在损失函数中引入经典本构方程残差,为模型提供物理约束。同时,针对实验中多个制备参数作为特征时出现的稀疏性和分散性问题,本文使用注意力特征融合方法进行特征降维,进一步优化了模型效果。此外,本文使用条件变分自编码器(CVAE)对实验数据进行反向建模,实现了从特定流变学性质数据生成实验制备参数的逆向设计,与PINN形成正逆向联合建模框架,为高分子材料的设计提供了有效的辅助支持工具。
\keywordsCN{流变学;本构方程;门控循环单元;物理信息神经网络;条件变分自编码器}

\chapter{Abstract}
Rheology is the science that studies the deformation and flow of matter, and rheological constitutive equations, as one of its core components, are mathematical models that describe the deformation and flow behaviors of materials under external forces. Traditional methods for obtaining and studying constitutive equations primarily rely on experimental measurements, theoretical derivation, and numerical simulation; however, these methods are often costly, time-consuming, and lack generalizability, which makes it challenging to meet the demands of modern materials science for efficient and accurate modeling.

In recent years, data-driven machine learning approaches have been applied to the construction of rheological constitutive equations, offering new avenues to overcome the limitations of traditional methods. Nonetheless, current machine learning-based constitutive modeling methods still suffer from insufficient physical constraints, inadequate consideration of the historical dependency inherent in rheological constitutive equations, feature sparsity, and challenges in reverse engineering the processing parameters.

To address these issues, this study investigates two modeling directions: time-domain rheological data modeling and frequency-domain data modeling. For time-domain stress–strain data, an innovative approach employing a physics-informed gated recurrent unit (PI-GRU) is proposed for predicting time-varying constitutive data. The gated mechanism of the GRU effectively captures the memory effects and strain-history dependency in rheology, while the inclusion of a physical residual term in the loss function imposes physical constraints. Compared with traditional deep feedforward neural networks, the PI-GRU demonstrates superior predictive performance and generalization ability on both numerical simulation data derived from classical constitutive equations and experimental rheological data.

For frequency-domain data, a physics-informed neural network (PINN) is employed to model the relationship between the material processing parameters and frequency-domain properties, such as storage modulus, loss modulus, and loss tangent. By incorporating the residual of the classical constitutive equation into the loss function, physical constraints are imposed on the model. Furthermore, to address the sparsity and dispersion issues encountered when multiple processing parameters are used as features in experiments, an attention-based feature fusion method is applied for dimensionality reduction, which further optimizes the model performance. In addition, a conditional variational autoencoder (CVAE) is used for inverse modeling of the experimental data, enabling the reverse design of processing parameters from specific rheological property data. Together with the PINN, this forms a forward–inverse joint modeling framework that serves as an effective tool for the design of polymer materials.
\keywordsEN{Rheology; Constitutive Equations; Gated Recurrent Unit; Physics-Informed Neural Networks; Conditional Variational Autoencoder}