\chapter{摘\texorpdfstring{\quad}{}要}
流变学是研究物质变形和流动的科学,流变学本构方程作为其核心内容之一,是描述材料在外力作用下变形和流动行为的数学模型。传统本构方程的获取和研究方法主要依赖实验测定、理论推导和数值模拟,这些方法存在成本高、耗时长、难以泛化等问题,难以满足现代材料科学对高效、精确建模的需求。

近年来,基于数据驱动的机器学习方法开始应用于流变学本构方程的构建研究中,为解决传统方法的局限性提供了新思路。然而,当前的机器学习本构建模方法仍存在物理约束不足、未充分考虑流变学本构方程的历史依赖性、特征稀疏以及难以反推制备参数等问题。

针对上述问题,本文从时间域流变数据建模和频域数据建模两个方向展开研究。在时间域应力应变数据方面,本文创新性地提出物理信息门控循环单元(Physics-Informed Gated Recurrent Unit, PI-GRU)进行时变本构数据的预测建模,利用GRU的门控机制有效捕捉流变学中的记忆效应和应变历史依赖性,同时将物理残差引入损失函数,增强模型的物理约束。相比传统深度前馈神经网络,PI-GRU在经典本构方程的数值模拟数据和真实流变学实验数据上均展现出更优的预测精度和泛化能力。

在频域数据方面,本文采用物理信息神经网络(Physics-Informed Neural Networks, PINN)对材料的制备参数与储存模量、损耗模量、损耗角正切等频域特性关系进行正向建模,通过在损失函数中融入经典本构方程残差,为模型提供物理约束。针对实验中多个制备参数作为特征时出现的稀疏性和分散性问题,本文引入注意力特征融合方法进行特征降维,进一步提升了模型性能。此外,本文应用条件变分自编码器(Conditional Variational Autoencoder, CVAE)对实验数据进行反向建模,实现了从特定流变学性质数据生成实验制备参数的逆向设计。通过构建PINN-CVAE正逆向联合建模框架,为高分子材料的设计提供了高效的辅助支持工具。
\keywordsCN{流变学;本构方程;循环神经网络;物理信息神经网络}

\chapter{Abstract}
Rheology is the scientific study of the deformation and flow of matter. Rheological constitutive equations serve as mathematical models that describe material behavior under external forces. Traditional approaches to developing these equations rely primarily on experimental measurements, theoretical derivation, and numerical simulation. However, these methods face significant limitations including high costs, time-consuming processes, and poor generalizability, making them inadequate for modern materials science demands.

In recent years, data-driven machine learning methods have emerged as promising alternatives for constructing rheological constitutive equations, providing new approaches to overcome the limitations of traditional methods. However, current machine learning approaches still face several challenges, including insufficient physical constraints, inadequate consideration of history-dependent properties in rheological systems, feature sparsity issues, and difficulties in reverse-engineering processing parameters.

To address these challenges, this dissertation explores both time-domain and frequency-domain rheological data modeling approaches. For time-domain stress-strain data, we propose a novel Physics-Informed Gated Recurrent Unit (PI-GRU) framework that effectively captures memory effects and strain-history dependencies through its gating mechanism, while incorporating physical residuals in the loss function to enhance physical constraints. Compared to traditional deep feedforward neural networks, our PI-GRU demonstrates superior predictive accuracy and generalization capabilities on both classical constitutive equation simulations and real rheological experimental data.

For frequency-domain modeling, we employ Physics-Informed Neural Networks (PINNs) to characterize relationships between material processing parameters and frequency-domain properties, including storage modulus, loss modulus, and loss tangent. We incorporate classical constitutive equation residuals as physical constraints to enhance model reliability. To address the sparsity and dispersion challenges associated with multiple processing parameters, we implement an attention-based feature fusion approach for dimensionality reduction, significantly enhancing model performance. Additionally, we develop a Conditional Variational Autoencoder (CVAE) for inverse modeling, enabling reverse design from rheological properties to processing parameters. The resulting PINN-CVAE bidirectional modeling framework provides an efficient tool for polymer material design and optimization.
\keywordsEN{Rheology; Constitutive Equation; Recurrent Neural Network; Physics-Informed Neural Network}