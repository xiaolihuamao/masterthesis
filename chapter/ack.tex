\chapter{致\texorpdfstring{\quad}{}谢}
南柯一梦,三载华工。时光如雨丝轻落,花影斑驳,往昔的点滴在心头悄然流转。回望这段旅程,我或许并非传统意义上的“好学生”,但人生的意义,常常藏在那些不被定义的时刻里。幸运的是,在这段不算璀璨的研究生岁月里,我遇见了许多温暖的人,他们的陪伴与点拨,让这段旅途成为我生命中不可磨灭的印记。

首先,衷心感谢我的导师,周嘉嘉老师。与周老师的相识虽短,却深刻如春风化雨。有限的时光里,老师对我的学术研究给予了极大的帮助,从论文选题到中期检查,再到最后的预答辩,每一步都耐心指导。他治学严谨,热忱专注,在他身上我看见了学者的风骨,也学会了如何在迷茫中寻找方向。

其次,感谢师门的每一位同学。感谢袁爽、操能杰、周香归等师兄师姐在学术上的悉心帮助。正是因为有这样一个团结、优秀的课题组,我才能在学术的森林中不至于迷失。同行者的陪伴,让孤独的求索变得温暖而有力。

再者,感激我的家人和朋友们。你们在我最迷茫时给予我安慰,在我最无助时给予我支持,在我最需要时给予我鼓励。你们的爱与陪伴,是我心灵深处最坚实的依靠。

最后,我要感谢那个永不言弃的自己,我始终坚信即使是西西弗斯,也能在日复一日的推石中找到属于自己的意义。
\begin{flushright}
  傅星源\\
  2025年8月20日于华园
\end{flushright}

