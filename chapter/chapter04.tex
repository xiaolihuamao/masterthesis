\chapter{基于流变学真实实验数据的物理信息-生成式混合建模方法研究}
% 引言 引出第四章内容
\todo[size=\small]{sff}
\section{引言}
流变学作为研究复杂流体与软物质力学响应的核心学科,其本构建模始终面临着实验数据与理论模型间的鸿沟。传统建模方法多依赖于理想化假设下的数值模拟,虽能构建出形式优美的本构方程,却难以准确捕捉真实材料在复杂工况下的非线性响应特征。近年来,随着高分子功能材料在柔性电子、生物医学等领域的突破性应用,材料体系呈现出多组分耦合、多尺度结构并存的特性,这使得基于第一性原理的建模方法在实验数据拟合精度与逆向设计效率方面遭遇双重瓶颈。针对这一科学难题,本研究提出物理信息驱动与生成式建模相融合的创新范式,通过深度整合流变学真实实验数据的多模态特征与物理守恒定律的强约束机制,构建起兼具预测精度与可解释性的智能建模框架。

当前流变学本构建模领域存在两大核心矛盾:其一,传统物理模型在描述真实材料复杂流变行为时,常因过度简化导致预测偏差累积,而纯数据驱动的黑箱模型虽能实现高精度拟合,却丧失了物理可解释性这一流变学研究的本质诉求;其二,材料逆向设计过程中,基于试错法的实验优化模式耗费大量资源,而现有生成模型在流变学参数空间的可控生成方面缺乏物理约束,导致生成结果常偏离热力学可行域。本研究通过分析高分子嵌段共聚物(PBA)及其功能梯度材料(PFGs)的流变测试数据集,发现其储存模量-频率曲线在时温等效原理下呈现典型的分段幂律特征,这为构建物理信息嵌入的神经网络提供了天然的约束条件。与此同时,多组分材料体系的流变指纹具有高维参数空间中的低维流形特性,这为变分自编码器的潜空间建模创造了可行性基础。

本研究创新性地将物理信息神经网络(PINN)与条件变分自编码器(CVAE)进行耦合,建立双向建模通道:在正向建模路径中,通过将质量守恒、动量守恒等物理定律编码为PINN的软约束条件,有效解决了小样本实验数据下的过拟合问题;在逆向设计路径中,利用CVAE的潜空间探索能力,结合流变响应的物理可行性验证模块,实现了从目标流变特性到材料制备参数的可控映射。这种混合建模策略不仅突破了传统方法在数据-物理融合层面的技术壁垒,更重要的是构建了闭环的材料设计-验证工作流,为智能流变学奠定了方法论基础。本章后续内容将系统阐述该混合建模方法的数据处理流程、网络架构创新以及验证实验结果,重点揭示物理约束与数据驱动机制的协同作用规律,为拓展智能建模在复杂流体研究中的应用边界提供新的理论工具。

% 实验设计  描述本章的实验过程,介绍黄金的工作,模型相关公式等等
\section{实验设计}
\subsection{PINN数据预处理}
\subsection{PINN模型训练}
\subsection{PINN模型测试}
\subsection{CVAE反向建模训练}
\subsection{CVAE反向建模测试}

% 结果与讨论 全面的数据展开
\section{结果与讨论}
\subsection{低保真数据拟合}
\subsection{单PBA流体本构建模}
\subsection{单PBA嵌入的PFGs本构建模}
\subsection{多组分PBA嵌入的PFGs本构建模}
\subsection{CVAE组分预测}

% 本章小结 总结与展望
\section{本章小结}