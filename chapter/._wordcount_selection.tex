通过建立PINN-CVAE的联合建模,可以建立一个自适应增强的流变学机器学习系统。该系统的工作流程如下:首先,PINN模型通过物理约束和特征融合机制对流变学数据进行正向建模,建立组分特征到流变学性质的映射关系。在实际应用过程中,通过真实实验获得的新数据可以不断补充到训练集中,进一步提升PINN模型的预测精度。与此同时,CVAE模型负责从目标流变学性质反向生成可能的组分配比方案,为材料设计提供多样化的参考。这些由CVAE生成的组分方案可以输入到PINN模型中进行快速评估和筛选,通过比较预测曲线与目标曲线的吻合度,筛选出最具潜力的组分配比进行实验验证,从而大大减轻实验负担。最后,实验验证结果又可以作为新的训练数据反馈给两个模型,形成一个不断优化的闭环系统。这种联合建模方法充分发挥了两种模型的优势,既保证了预测的物理合理性,又提供了材料设计的多样化方案,同时通过数据驱动和实验验证的结合,实现了系统性能的持续提升。