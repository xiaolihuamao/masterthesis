\begin{enumerate}[topsep = 0 pt, itemsep= 0 pt, parsep=0pt, partopsep=0pt, leftmargin=44pt, itemindent=0pt, labelsep=6pt, label=(\arabic*)]
  \item 针对黏弹性流体的非线性本构关系,本文采用的GRU模型在处理长时间尺度的非线性本构关系时存在长期依赖问题。在实验条件允许且具备充足高质量流变学数据的情况下,可以考虑引入基于自注意力机制的Transformer架构。该架构通过多头注意力机制和位置编码,能够更好地处理长序列数据,有望捕捉材料在更长时间尺度下的流变学特性,尤其是在应力松弛和蠕变等长期行为的建模中具有潜在优势。

  \item 可以考虑引入多尺度建模方法,将分子动力学模拟与宏观流变学行为建模相结合,使用多模态数据来训练模型,可以从理论和实验两个角度来提高模型的泛化能力。
\end{enumerate}
