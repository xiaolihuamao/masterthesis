%\chapter{结\texorpdfstring{\quad}{}论}
\chapter{总结与展望}

\section*{研究工作总结}
本文以流变学本构建模为研究主题,系统探索了深度学习方法在流变学本构建模中的应用前景。研究内容主要包含两个方面:

第一个方面聚焦于深度学习模型结构对流变学本构建模的适用性研究。考虑到黏弹性材料流变特性中的应变历史依赖性,本文创新性地引入了天然适合处理时序数据的循环神经网络,特别是采用了GRU,并与PINN的方法进行结合,构建了PI-GRU模型。

第二个方面是在真实的材料制备与流变学测试场景下,使用PINN和CVAE两种方法对流变学本构模型分别进行正逆向训练。通过引入注意力特征融合机制,探讨了在实验数据有限、特征信息稀疏的情况下使用PINN建模以预测材料流变学性质的可行性。而CVAE的引入则探讨了另一个材料科研中的重要问题——如何通过期望的流变学性质反推出材料的制备参数来辅助材料设计。最终,本文构建了基于PINN-CVAE的联合建模框架,通过将PINN的正向预测能力与CVAE的反向生成能力相结合,形成了一个自适应增强的流变学机器学习系统。

在第一部分研究中,本文首先通过数值模拟生成了包括Herschel-Bulkley模型、Maxwell模型、Doi-Edwards模型和Giesekus模型在内的多种典型流变学模型的数据。这些模型涵盖了从简单到复杂、从线性到非线性的不同类型本构关系,为深度学习方法的验证提供了全面的数据基础。研究结果表明,GRU的门控机制在处理具有时间依赖性的流变学数据方面展现出显著优势,相较于传统前馈神经网络结构的DNN模型在预测精度和泛化能力上有明显提升。这一发现对于改善流变学时序数据的处理方法具有重要的科学意义,为解决传统本构模型在复杂流变行为描述中的局限性提供了新思路。

本研究创新性地在GRU基础上引入物理本构残差约束,构建了PI-GRU模型,成功验证了该模型具有强大的跨测试模式预测泛化能力。PI-GRU模型不仅能够通过振荡剪切训练数据准确预测不同条件下的振荡剪切和稳态剪切行为,还能在一定程度上利用线性黏弹性区间的SAOS数据推断非线性黏弹性区间的LAOS数据特性。这对于减少复杂流变测试的实验成本具有实际应用价值。随后,本文将PI-GRU模型应用于真实黏弹性材料的流变学数据建模,验证结果表明该模型在实际测试环境中依然保持良好的预测性能,其预测精度和泛化能力均优于对照模型,为材料流变特性的快速准确表征提供了有效工具。

在第二部分研究中,本文以高分子流变学数据建模及组分反演为研究对象,构建了PINN-CVAE正逆向联合建模框架。在正向建模方面,研究发现普通PINN模型虽然相比传统DNN具有更好的泛化预测效果,但在特征数量较多且分布稀疏的场景下表现一般。为此,本文提出了PINN-HP和PINN-AFF两种改进方案。通过预设和学习输入特征之间的物理关联,有效改善了模型对复杂非线性关系的捕捉能力。实验结果表明,PINN-AFF模型在预测准确性和稳定性上均明显优于传统DNN和普通PINN模型。

在逆向建模方面,本文利用CVAE模型从目标流变学参数反向生成材料组分配比。结果表明,生成的分子量参数和组分含量参数均呈现正态分布特征,最大误差控制在10\%以内。通过将PINN的正向预测能力与CVAE的反向生成能力相结合,本研究构建了一个自适应增强的流变学机器学习系统,为高分子材料的智能设计提供了新思路。

\section*{主要研究贡献}
本文的研究工作为深度学习方法在流变学本构建模中的应用提供了创新性的探索,主要贡献如下:

\begin{enumerate}[topsep = 0 pt, itemsep= 0 pt, parsep=0pt, partopsep=0pt, leftmargin=44pt, itemindent=0pt, labelsep=6pt, label=(\arabic*)]
  \item 在时间域流变学数据(应力、应变率-应力)的研究中,创新性地引入了更擅长处理时序数据的GRU作为基础模型,并结合物理约束构建了PI-GRU模型,显著提升了流变学本构建模的精度和泛化能力;
  \item 在频率域流变学数据(频率-储存模量、损耗模量、损耗角正切)的研究中,提出了注意力特征融合的PINN优化方案,有效提高了模型对复杂非线性关系的表征能力;
  \item 首次尝试了基于CVAE的逆向训练框架,实现了从目标流变性能到材料配方的反向设计,为材料智能设计提供了新方法。
\end{enumerate}

\section*{未来研究展望}
基于本文的研究成果,未来可在以下几个方向进行深入研究:

\begin{enumerate}[topsep = 0 pt, itemsep= 0 pt, parsep=0pt, partopsep=0pt, leftmargin=44pt, itemindent=0pt, labelsep=6pt, label=(\arabic*)]
  \item 引入Transformer架构改善长时间尺度建模:针对黏弹性流体的非线性本构关系,本文采用的GRU模型在处理长时间尺度的非线性本构关系时存在长期依赖问题。在实验条件允许且具备充足高质量流变学数据的情况下,可以考虑引入基于自注意力机制的Transformer架构。该架构通过多头注意力机制和位置编码,能够更好地处理长序列数据,有望捕捉材料在更长时间尺度下的流变学特性,尤其是在应力松弛和蠕变等长期行为的建模中具有潜在优势。此外,还可以探索将Transformer与物理约束相结合的方法,通过在注意力层中引入物理先验知识来增强模型对材料本构关系的理解。

  \item 发展多尺度建模方法:可以考虑引入多尺度建模方法,将分子动力学模拟与宏观流变学行为建模相结合。通过分子动力学模拟获取材料微观结构演化信息,结合宏观流变实验数据,构建跨尺度的深度学习模型。这种多模态数据训练方式不仅可以从理论和实验两个角度提高模型的泛化能力,还能帮助揭示材料微观结构与宏观流变性质之间的关联机制。同时,可以探索使用图神经网络等新型架构来处理分子结构数据,进一步提升模型对材料结构-性能关系的表征能力。

  \item 构建端到端的材料配方优化系统:在PINN-CVAE联合建模框架的基础上,可以进一步探索引入强化学习方法,构建端到端的材料配方优化系统。通过将PINN的物理约束预测能力、CVAE的多模态生成能力与强化学习的决策优化能力相结合,可以实现从目标性能出发,自动搜索和优化材料配方的智能设计流程。这种方法有望大幅提高材料开发效率,减少试错成本。同时,还可以考虑在优化过程中引入多目标约束,平衡材料性能、成本和工艺等多个维度的需求。
\end{enumerate}

