%\chapter{结\texorpdfstring{\quad}{}论}
\chapter{总结与展望}
本文以流变学本构建模为研究对象,系统探索了深度学习方法在流变学本构建模中的应用前景。研究内容主要包含两个方面:第一个方面聚焦于深度学习模型结构对流变学本构建模的适用性研究。考虑到黏弹性材料流变特性中的应变历史依赖性,本文创新性地引入了天然适合处理时序数据的循环神经网络,特别是采用了门控循环单元(GRU),并与传统深度神经网络(DNN)进行了系统对比分析。第二个方面是在真实实验数据上,使用物理信息神经网络(PINN)和条件变分自编码器(CVAE)两种方法对流变学本构模型分别进行正逆向训练,通过引入注意力特征融合机制,探讨了在实验数据有限,特征信息稀疏的情况下使用PINN建模以预测材料流变学性质的可行性。而CVAE的引入则探讨了另一个材料科研中的重要问题,即如何通过期望的流变学性质反推出材料的制备参数来辅助材料设计。

在第一部分的研究工作中,本文采用了数值模拟与深度学习相结合的方式。首先通过数值模拟生成了包括Herschel-Bulkley模型、Maxwell模型、Doi-Edwards模型和Giesekus模型在内的多种典型流变学模型的数据。这些模型涵盖了从简单到复杂、从线性到非线性的不同类型本构关系,为深度学习方法的验证提供了全面的数据基础。在深度学习方面,本文重点比较了传统深度神经网络(DNN)和门控循环单元(GRU)两种算法的建模效果。研究发现,GRU算法凭借其独特的门控机制,在处理具有时间依赖性的流变学数据时展现出显著优势。在模型泛化能力方面,本文通过交变应变预测和线性应变预测两类任务进行了验证。结果显示,GRU模型表现出更强的泛化能力,能够准确预测不同应变条件下的材料响应。这一发现对于实际应用具有重要意义,说明深度学习模型不仅能够进行简单的数据拟合,还能够捕捉材料本构关系的本质特征。然而,研究也发现GRU模型的优势是以更高的计算成本为代价的,其训练时间和资源消耗都显著高于DNN。

第二部分的研究工作中,本文使用一类黏弹性凝胶材料的实验数据作为研究对象,使用PINN方法对流变学本构模型进行正向训练。研究发现,PINN模型在预测黏弹性凝胶材料的流变学性质时,表现出较好的泛化能力,优于纯数据驱动的DNN模型,这得益于PINN模型能够从数据中学习到流变学本构模型的物理信息,从而在一定程度上克服了数据不足的问题。而面对更为复杂的但是特征较为稀疏的预测场景,例如多组分分子量的原材料按照一定比例投入制备黏弹性凝胶材料时,PINN模型在预测时表现出了局限性,本文通过引入注意力特征融合机制,很好地解决了这一问题,研究结果表明,注意力特征融合机制能够显著提高PINN模型的预测精度,从而为流变学本构模型的正向训练提供了一种优化方案。在逆向训练方面,本文使用CVAE模型对黏弹性凝胶材料的制备参数进行预测,研究结果表明,CVAE模型在预测时表现出了较好的泛化能力,能够较准确预测不同制备参数下的材料流变学性质,最大误差在10\%以内,从而为黏弹性凝胶材料的制备参数设计提供了一种新的思路。


总的来说,本文的研究工作为深度学习方法在流变学本构建模中的应用提供了创新性的探索。在模拟数据的研究中,本文创新性地引入了更擅长处理时序数据的GRU模型,在实验数据的研究中,本文在传统PINN的基础上提出了注意力特征融合的优化方案,并首次尝试了基于CVAE的逆向训练来辅助材料设计。基于本文的研究成果,未来还可以在以下几个方向进行深入研究:

\begin{enumerate}[topsep = 0 pt, itemsep= 0 pt, parsep=0pt, partopsep=0pt, leftmargin=44pt, itemindent=0pt, labelsep=6pt, label=(\arabic*)]
  \item 针对黏弹性流体的非线性本构关系,本文采用的GRU模型在处理长时间尺度的非线性本构关系时存在长期依赖问题。在实验条件允许且具备充足高质量流变学数据的情况下,可以考虑引入基于自注意力机制的Transformer架构。该架构通过多头注意力机制和位置编码,能够更好地处理长序列数据,有望捕捉材料在更长时间尺度下的流变学特性,尤其是在应力松弛和蠕变等长期行为的建模中具有潜在优势。

  \item 可以探索将物理信息神经网络(PINN)与循环神经网络架构进行深度融合,构建一个端到端的混合模型。具体而言,可以将GRU的时序建模能力与PINN的物理约束相结合,使模型既能准确捕捉本构关系中的时间依赖性,又能保持物理一致性。这种混合架构在处理有限实验数据时,可以通过物理约束来增强模型的泛化能力,同时保持对材料动态行为的准确描述。此外,还可以考虑引入多尺度建模方法,将分子动力学模拟与宏观流变学行为建模相结合,使用多模态数据来训练模型,可以从理论和实验两个角度来提高模型的泛化能力。
\end{enumerate}

