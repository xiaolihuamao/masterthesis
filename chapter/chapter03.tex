
\chapter{常用环境及参考文献设置}
强烈建议在使用公式、表格、定理环境时进行百度,没必要研究各种用法,只需要知道自己需要什么。因本人的论文所用表格较少,因而对表格不是很熟悉,本章对表格的介绍相应的较少。本章仅介绍本人在论文撰写过程中常用的环境以及参考文献设置。

\section{图}
图的导入需要提前准备好图片文件,最好是.png、.eps、.pdf或.jpg文件。另外,如果是从matlab导出图片文件,可使用print函数或手动导出,print函数的使用可参考ICGNC2020plot.m以及PlotToFileColorPDF.m文件等。手动导出(matlab的figure界面的“文件”->“导出设置”设置好大小、分辨率和线宽等然后点击“应用于图窗”)主要用于观察效果,可设置某种样式名称后保存该样式,下次使用时加载,具体可百度“matlab导出高清图片”。需要特别注意的是一定要1:1导入matlab生成的图片,并且图中文字设置好字体字号。否则缩放之后,图片的字号就变了,盲审老师一眼就能看出来字号不对,就很麻烦。这就是为什么要在matlab点击“应用于图窗“进行预览,观测效果后再1:1使用图片。

% 3.2表================================================
\section{表}
本节仅展示使用常见的三线表
\begin{table}
	\TableBicaption{\label{TDF_para}涵道模型参数}{Parameters of Ducted Fan Model}  % 中英文标题
	\centering
	\small
	\begin{tabularx}{\textwidth}{XXXX}  % 使用 tabularx 环境,均匀分布列宽
		\Xhline{1.5pt}
		参数符号       & 数值                 & 参数符号  & 数值                 \tabularnewline
		\Xhline{0.5pt}  % 表头下方线
		$I_x$      & $054593$           & $I_y$ & $0.017045         $ \tabularnewline
		$l_1$      & $0.0808\,\text{m}$ & $l_2$ & $0.175\,\text{m}  $ \tabularnewline
		$l_4$      & $0.2415\,\text{m}$ & $l_5$ & $0.1085\,\text{m} $ \tabularnewline
		$l_\sigma$ & $xdf$              & $df$  & 扫描电镜 \tabularnewline
		\Xhline{1.5pt}
	\end{tabularx}
\end{table}

\begin{table}
	\TableBicaption{\label{TDF_para2}涵道模型参数}{Parameters of Ducted Fan Model}  % 中英文标题
	\centering
	\small
	\begin{tabularx}{\textwidth}{XXXX}  % 使用 tabularx 环境,均匀分布列宽
		\Xhline{1.5pt}
		参数符号       & 数值                 & 参数符号  & 数值                 \tabularnewline
		\Xhline{0.5pt}  % 表头下方线
		$I_x$      & $054593$           & $I_y$ & $0.017045         $ \tabularnewline
		$l_1$      & $0.0808\,\text{m}$ & $l_2$ & $0.175\,\text{m}  $ \tabularnewline
		$l_4$      & $0.2415\,\text{m}$ & $l_5$ & $0.1085\,\text{m} $ \tabularnewline
		$l_\sigma$ & $xdf$              & $df$  & 扫描电镜 \tabularnewline
		\Xhline{1.5pt}
	\end{tabularx}
\end{table}


\section{公式}
除了前面讲行内公式,常用的还有行间公式。公式中的数学符号可自行百度,本章仅介绍常用的几种公式环境。

单独成行的行间公式在 \LaTeX{} 里由equation 环境包裹。equation 环境为公式自动生成一个编号,这个编号可以用\textbackslash{}label 和\textbackslash{}ref 生成交叉引用,amsmath 宏包的\textbackslash{}eqref 可为引用自动加上圆括号;如式\eqref{eq_1}所示。
\begin{lstlisting}
\begin{equation}
	a+b=c	\label{eq_1}
\end{equation}
\end{lstlisting}
\begin{equation}
	a+b=c	\label{eq_1}
\end{equation}
若不需要编号则加星号,改为
\begin{lstlisting}
\begin{equation*}
	a+b=c
\end{equation*}
\end{lstlisting}
其他环境类似。当使用 \texttt\$ 开启行内公式输入,或是使用{equation} 环境时,\LaTeX\ 就进入了数学模式。
数学模式相比于文本模式有以下特点:
\begin{enumerate}
	\item 数学模式中输入的空格被忽略。数学符号的间距默认由符号的性质(关系符号、运算符等)决定。
	      需要人为引入间距时,使用 \textbackslash{}{quad} 和 \textbackslash{}{qquad} 等命令。
	\item {不允许有空行(分段)}。行间公式中也无法用 $ \verb|\\|$命令手动换行。排版多行公式需要用到 其他各种环境。
	\item 所有的字母被当作数学公式中的变量处理,字母间距与文本模式不一致,也无法生成单词之间的空格。
	      如果想在数学公式中输入正体的文本,简单情况下可用 \textbackslash{}{mathrm} 命令。
	      或者用 {amsmath} 提供的 \textbackslash{}{text} 命令(仅适合在公式中穿插少量文字。如果你的情况正好相反,需要在许多文字中穿插使用公式,则应该像正常的行内公式那样用,而不是滥用 \textbackslash{}{text} 命令)。
\end{enumerate}

实际上更常用的的是多行公式,不需要对齐的公式组可以使用gather环境,需要对齐的公式组用align 环境。
长公式内可用$ \verb|\\|$ 换行。

这是一个正文公式:
Mn$^{12}$

如果需要罗列一系列公式,并令其按照等号对齐,可用align 环境,它将公式用\& 隔为两部分并对齐。分隔符通常放在等号左边:
\begin{lstlisting}
\begin{align}
	a & = b + c \\
	& = d + e
\end{align}
\end{lstlisting}
\begin{align}
	a & = b + c \\
	  & = d + e
\end{align}
align 环境会给每行公式都编号。

如果不需要按等号对齐,只需罗列数个公式,可用gather环境:
\begin{lstlisting}
\begin{gather}
	a  = b + c \notag \\
	f = d + e 
\end{gather}
\end{lstlisting}
\begin{gather}
	a  = b + c \notag  \\
	f = d + e
\end{gather}
gather 环境同样会给每行公式都编号,如果某行不需要编号可在行末用\textbackslash{}notag 仅去掉某行的编号。

align 和gather 有对应的不带编号的版本align* 和gather*。

另一个常见的需求是将多个公式组在一起公用一个编号,编号位于公式的居中位置。为此,
amsmath 宏包提供了诸如aligned、gathered 等环境,与equation 环境套用。以-ed 结尾的
环境用法与前一节不以-ed 结尾的环境用法一一对应。我们仅以aligned 举例:
\begin{lstlisting}
\begin{equation}
	\begin{aligned}
		a &= b + c \\
		d &= e + f + g \\
		h + i &= j + k \\
		l + m &= n
	\end{aligned}
\end{equation}
\end{lstlisting}
\begin{equation}
	\begin{aligned}
		a     & = b + c+\sigma_{121234} \\
		d     & = e + f + g             \\
		h + i & = j + k                 \\
		l + m & = n
	\end{aligned}
\end{equation}
split 环境和aligned 环境用法类似,也用于和equation 环境套用,区别是split 只能
将每行的一个公式分两栏,aligned 允许每行多个公式多栏。

分段函数通常用amsmath 宏包提供的cases 环境,可参考文献

amsmath 宏包还直接提供了多种排版矩阵的环境,包括不带定界符的matrix,以及带各种定界符的矩阵pmatrix、bmatrix、Bmatrix、vmatrix、Vmatrix。
其中中括号版的bmatrix最常用。这些矩阵环境需要在公式中使用,比如 gather 环境。
\begin{lstlisting}
\begin{gather}
	\boldsymbol{A}= \begin{bmatrix}
		x_{11} & x_{12} & \ldots & x_{1n} \\
		x_{21} & x_{22} & \ldots & x_{2n} \\
		\vdots & \vdots & \ddots & \vdots \\
		x_{n1} & x_{n2} & \ldots & x_{nn}
	\end{bmatrix}
\end{gather}
\end{lstlisting}
这是一个正文公式
\begin{equation}
	\begin{aligned}
		f(x) & = x^2 + 1, & \text{如果 } x \geq 0 \\
		     & = -x^2,    & \text{如果 } x < 0
	\end{aligned}
\end{equation}

\begin{gather}
	\boldsymbol{A}= \begin{bmatrix}
		x_{11} & x_{12} & \ldots & x_{1n} \\
		x_{21} & x_{22} & \ldots & x_{2n} \\
		\vdots & \vdots & \ddots & \vdots \\
		x_{n1} & x_{n2} & \ldots & x_{nn}
	\end{bmatrix}
\end{gather}
其中矩阵/向量加粗使用\textbackslash{}boldsymbol\{\}命令,\textbackslash{}bm\{\}命令和unicode-math包有兼容性问题。另外还可以使用array环境排版矩阵,类似tabular环境,用$ \verb|\\|$ 和\& 用来分隔行和列,这里不再赘述。
\begin{lstlisting}
\begin{array }[外部对齐tcb]{列对齐lcr}
	行列内容
\end{array}
\end{lstlisting}

另外注意排版分式时,有两种方法:\textbackslash{}frac或者\textbackslash{}dfrac,效果分别为$ \frac{1}{2} $和$ \dfrac{1}{2} $。以上介绍的数学环境中,空格可参考文献,例如常用\textbackslash{}quad。

需要局部更改字号时,可以使用tutorial文件夹lshort-zh-cn.pdf的5.1节进行更改,如加\textbackslash{}small使得字号小一号。
\section{定理}
在scutthesis.cls文件的最后,已经用\textbackslash{}newtheorem命令定义了几种定理环境,包括:定义、假设、定理、结论、引理、公理、推论、性质等等,统称定理环境,关于\textbackslash{}newtheorem的用法,可参考或自行百度。要下面提供几个例子,在横线之间的深色区域是代码,效果在相应下方表示:
\begin{lstlisting}
\begin{assumption}
	加权矩阵${{\boldsymbol{W}}_{1}}$和 ${{\boldsymbol{W}}_{2}}$ 是对称矩阵,且$ {{\boldsymbol{W}}_{2}}$非奇异。	\label{assum_dca1}
\end{assumption}
\end{lstlisting}
\begin{assumption}
	加权矩阵${{\boldsymbol{W}}_{1}}$和 ${{\boldsymbol{W}}_{2}}$ 是对称矩阵,且$ {{\boldsymbol{W}}_{2}}$非奇异。	\label{assum_dca1}
\end{assumption}

定理用法和假设类似:
\begin{lstlisting}
\begin{theorem}
	如果假设\ref{assum_dca1}成立,$\boldsymbol{F}$满足式\eqref{eq_F}的定义,且${{\boldsymbol{W}}_{1}}$非奇异,则有$0\le e \left( \boldsymbol{F} \right) < 1$,其中$e \left( \boldsymbol{F} \right)$是 $\boldsymbol{F}$的特征值。	\label{the_dca2}
\end{theorem}
\end{lstlisting}
\begin{theorem}
	如果假设\ref{assum_dca1}成立,$\boldsymbol{F}$满足上式的定义,且${{\boldsymbol{W}}_{1}}$非奇异,则有$0\le e \left( \boldsymbol{F} \right) < 1$,其中$e \left( \boldsymbol{F} \right)$是 $\boldsymbol{F}$的特征值。	\label{the_dca2}
\end{theorem}
\begin{remark}
	定理环境的编号可自定义,但通常不需要再进行设置,因为模板文件scutthesis.cls文件已经定义好。
\end{remark}

---------------------------------------------------------

2022年5月更新:



amsthm 提供了 \textbackslash{}theoremstyle 命令支持定理格式的切换,在用 \textbackslash{}newtheorem 命令定义定 理环境之前使用。amsthm 预定义了三种格式用于 \textbackslash{}theoremstyle:plain 和 LATEX 原始的格式 一致;definition 使用粗体标签、正体内容;remark 使用斜体标签、正体内容。

以上部分在scutthesis.cls文件最后一部分设置。

---------------------------------------------------------

amsthm 还提供了一个 proof 环境用于排版定理的证明过程。proof 环境末尾自动加上一个证毕符号:
\begin{proof}
	显然有
	\[
		E=mc^2
	\]
	证毕
\end{proof}



\section{参考文献}

再次强调,使用其他参考文献管理软件的用户以及不使用任何软件的“裸奔”的用户不需要关注任何关于zetero的东西。
\begin{lstlisting}
	关于参考文献这块,很多同学有疑问。只有记住一点:不管用什么参考文献管理工具,最终目的是生成一个bib文件,bib文件里是特定格式的文献信息。bib文件当作文本打开,里面就是文献的元数据。
\end{lstlisting}


引用前手动加空格,如:


手写方括号 [6]。引用后面没空格。












