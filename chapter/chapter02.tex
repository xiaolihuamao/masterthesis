\chapter{模板简介}
%
与很多外文杂志社不同,大部分中文期刊都不提供\LaTeX{}模板给投稿者使用,也很少有学校给学生提供官方的毕业论文模板。目前github上的大部分模板都是由学生发起的非官方模板。在此感谢Shun Xu以及yecfly等人的工作,他们的无私贡献使得华南理工大学硕博士毕业论文也可以使用\LaTeX{}撰写。

本模板是直接修改前人的模板得到的,更详细的介绍可到下载。本章仅从用户的角度简要介绍模板的使用,而尽量避免涉及\LaTeX{}的模板制作细节(实际上是因为本人也不会)。正如我们使用手机并不需要了解麦克斯韦方程组,使用\LaTeX{}写作也无需了解模板是如何制作的。

\LaTeX{}的源代码保存在后缀名为.tex的文件中。当编写长篇文档时,例如当编写书籍、毕业论文时,单个源文件会使修改、校对变得十分困
难。将源文件分割成若干个文件,例如将每章内容单独写在一个文件中,会大大简化修改和校对
的工作。为方便,本文将scutthesis.tex文件称为主文件,而将abstract.tex、chapter0x.tex、conclusion.tex等文件称为章节文件。

。

\section{主文件}
scutthesis.tex文件相当于主函数,调用各章的内容。\LaTeX{}源代码以一个\textbackslash{}documentclass 命令作为开头,它指定了文档使用的文档类。文档类规定了\LaTeX{}源代码所要生成的文档的性质——普通文章、书籍、演示文稿、个人简
历等等。
其中class-name为文档类的名称,如\LaTeX{}提供的article, book, report,可在其基础上派
生的一些文档类或者有其它功能的一些文档类。\LaTeX{}提供的基础文档类见文







综上,论文撰写只需要将自己的文本(包含行内公式)放到相应的章节处,并添加行间公式、图表环境并填写图表即可。行间公式、图表将在下一章介绍。

